\documentclass{article}
\usepackage[utf8]{inputenc}
\usepackage{dsfont}
\usepackage{amsmath}
\usepackage{nccmath}
\usepackage{scrextend}
\usepackage{amssymb}
\usepackage{multirow,array}
\usepackage[margin=1.5in]{geometry}
\usepackage{color,soul}
\begin{document}
\centering
\Large
\textbf{Convergence to a Convention in a 2x2 Coordination Game with Adaptive Learning}
\vskip0pt
Ethan Holdahl

\large
University of Oregon

\today

\vskip24pt

\centering
Coordination Game G
\vskip6pt
\begin{tabular}{cc|c|c|}
      & \multicolumn{1}{c}{} & \multicolumn{2}{c}{Player B}\\
      & \multicolumn{1}{c}{} & \multicolumn{1}{c}{0}  & \multicolumn{1}{c}{1} \\\cline{3-4}
      \multirow{2}*{Player A}  & 0 & ${a_{00},b_{00}}^*$ & $a_{01},b_{01}$ \\\cline{3-4}
      & 1 & $a_{10},b_{10}$ & ${a_{11},b_{11}}^*$ \\\cline{3-4}
    \end{tabular}
\vskip24pt

\raggedright

Lemma 1 will be used in the proof of Theorem 1.

\vskip12pt
Let G be a 2x2 coordination game, and let $P^{m,s,\epsilon}$ be adaptive learning with memory $m$, sample size $s$, and error rate $\epsilon$.
Denote the players A and B with strategies $\{0,1\}$ as depicted above where (0,0) and (1,1) constitute pure strategy Nash Equilibria. Define $h_{-i}^{t}=(x_{-i}^{t-m},..,x_{-i}^{t-1})$ as the most recent $m$ records of all players except player $i$ at time $t$. Define $R_i^{t}$ as the set of $s$ records sampled by player $i$ in period $t$ from $h_{-i}^{t}$. Define $BR_i^{t}$ as player $i$'s best response to $R_i^{t}$.

\vskip12pt

\textbf{Lemma 1.} Let G be a 2x2 coordination game, and let $P^{m,s,\epsilon}$ be adaptive learning with memory $m$, sample size $s$, and error rate $\epsilon$. If at any period $n$ $\exists$ a strategy profile $x^* = (x^*_1,x^*_2)$ that constitutes a Nash Equilibrium where $x^*_1 \in \{BR_1^{n}\}$ and $x^*_2 \in \{BR_2^{n}\}$ for players 1 and 2 then there exists a positive probability that each player $i$ plays $x^*_i$ as a best response for every period $e \geq n$.

\vskip12pt

\textbf{Proof of Lemma 1.} Assume at period $n$ $\exists$ an action $ x^* \in BR_A^{n} \cap BR_B^{n}$. I must show that there exists a positive probability that each player plays $x^*$ as a best reply for every period $e \geq n$.

\vskip12pt

I use proof by induction.

\vskip12pt

\textit{Base Step:} I show that if there exists an action $x^*$ that is a best response for both players in period $n$ then there exists a positive probability that both players play action $x^*$ in period $n$.

\vskip6pt

Clearly, if $x^*$ is a best response for both players then there exists a positive probability that both players play action $x^*$ in period $n$.

\vskip12pt

\textit{Inductive Step:} I show that $\forall e \geq n$ if action $x^*$ was played as a best response for both players in period $e$ then there exists a positive probability that action $x^*$ is played as a best response in period $e+1$.

\vskip6pt

Assume action $x^*$ was played as a best response for both players in period $e$. I must show that there exists a positive probability that action $x^*$ is played as a best response in period $e+1$.

\vskip6pt

Note that since $x_i^t \in \{0,1\}$ \hspace{4pt} $\forall t$, the proportion of times that player $j \neq i$ played 1 in $R_i^t$, the set of $s$ records in period $t$, is simply $\sum\limits_{r \in R_i^t} \cfrac{r}{s}$.

For example, if $R_i^t = (0,1,1,1,0,0,1,1)$ then then proportion of times player $j \neq i$ played 1 in player $i$'s sample, $R_i^t$ = $\sum\limits_{r \in R_i^t} \cfrac{r}{s} = \cfrac{5}{8} $

Without loss of generality assume that $x^{*}=1$. That means action 1 is a best response for each player $i \in \{A,B\}$ to $R_i^{e}$, the set of $s$ records sampled by player $i$ in period $e$.

\vskip6pt

Let $\alpha_i \in (0,1)$ be the smallest probability that Player $j \neq i$ plays action $1$ such that Player $i$'s best response is playing action $1$.

That means that for $i \in {A, B}$:

$$ (1) \sum\limits_{r \in R_i^{e}} \cfrac{r}{s} \geq \alpha_i$$

Now consider the set of $s$ records sampled by player $i$ in $e+1$: $R_i^{e+1}$.

\vskip6pt

In period $e+1$ each player $i$ samples $s$ records from $h_{-i}^{e+1}$, the most recent $m$
records of all players except player i at time t. Note that $|h_{-i}^{e} \cap (h_{-i}^{e+1})'|=1$, that is to say that there is only 1 record in $h_{-i}^{e}$ that is not in $h_{-i}^{e+1}$. Since $R_i^{e} \subseteq h_{-i}^{e}$ I know that $|R_i^{e} \cap (h_{-i}^{e+1})'| \leq 1$. That means there is at most 1 record in player $i$'s sample in period $e$ that is not able to be sampled in period $e+1$. This means that there is a positive probability that in period $e+1$, each player $i$ samples $s-1$ records from the set $h_{-i}^{e+1} \cap R_i^{e}$ and the most recent record, $x_{-i}^{e}$. Assume both players samples in period $e+1$ fit this criteria and define $c_{-i}^{e}=R_i^e \backslash R_i^{e+1}$, the record that was in the sample in period $e$ but not in period $e+1$ for player $i$.

\vskip6pt

So, $c_{-i}^{e}=R_i^e \backslash R_i^{e+1}$ and $x_{-i}^{e}=R_i^{e+1} \backslash R_i^{e}$. Consequently, I know that for each player $i$:

$$\frac{c_{-i}^{e}}{s} + \sum\limits_{r \in R_i^{e+1}} \cfrac{r}{s} = \frac{x_{-i}^{e}}{s} + \sum\limits_{r \in R_i^{e}} \cfrac{r}{s}$$

However, I know that $x_{-i}^{e}=1$ and $c_{-i}^{e} \in \{0,1\}$. So, $x_{-i}^{e} \geq c_{-i}^{e}$.

Adding to both sides I get:

$$\frac{c_{-i}^{e}}{s} + \frac{x_{-i}^{e}}{s} + \sum\limits_{r \in R_i^{e+1}} \cfrac{r}{s} \geq \frac{x_{-i}^{e}}{s} + \frac{c_{-i}^{e}}{s} + \sum\limits_{r \in R_i^{e}} \cfrac{r}{s}$$

So

$$(2) \sum\limits_{r \in R_i^{e+1}} \cfrac{r}{s} \geq \sum\limits_{r \in R_i^{e}} \cfrac{r}{s}$$

Using (1) and (2) by transitivity I get:

$$\sum\limits_{r \in R_i^{e+1}} \cfrac{r}{s} \geq \sum\limits_{r \in R_i^{e}} \cfrac{r}{s} \geq \alpha_i$$

So for each player $i \in \{A,B\}$ action 1 is a best response to the sample $R_i^{e+1}$. Since action 1 is a best response for both players in period $e+1$ there exists a positive probability that each plays action 1 in period $e+1$.

\vskip6pt

Since both the base case and the inductive step has been shown, by mathematical induction I have proven that if both players play $x^*$ as a best response in period $n$, then there exists a positive probability that both players play $x^*$ as a best reply for every period $e \geq n$. $\square$

\pagebreak

\textbf{Theorem 1.} Let G be a 2x2 coordination game, and let $P^{m,s,\epsilon}$ be adaptive learning with memory $m$, sample size $s$, and error rate $\epsilon$.

\vskip12pt

If $s < m$ then from any initial state, the unperturbed process $P^{m,s,0}$ converges with probability one to a convention and locks in.

\vskip24pt

\textbf{Proof of Theorem 1.} Define G as a 2x2 coordination game with adaptive learning where the possible actions for both players A and B are $\{0 ,1 \}$. Let memory $m \in \mathds{N}$, sample size $s \in \mathds{N}$ such that $s<m$, error rate $\epsilon = 0$ and let $h^t$ $= (x^{t-m+1},...,x^{t})$, be an arbitrary state at the end of period $t$. Let $\alpha \in (0,1)$ be the smallest probability that Player B plays action $1$ such that Player A's best response is playing action $1$. Likewise, Let $\beta \in (0,1)$ be the smallest probability that Player A plays action $1$ such that Player B's best response is playing action $1$. 

\vskip12pt

There exists a positive probability that both players sample the most recent set of $s$ records: $\{x^{t-s+1},...,x^{t}\}$ in period $t+1$. Assume this is the case.

\vskip12pt

In period $t+1$ the two players either

1) Share a best reply

or 

2) Do not share a best reply

I will show a convention can be reached with positive probability in both cases.

\textit{Case 1}: Both players share a best reply, $x^*$, in period $t+1$. In this case I can apply Lemma 1 which shows that there exists a positive probability that both players play action $x^*$ as a best reply for each period $e \geq t+1$. If this happens then after period $e=t+m$ the entire memory is filled with both players playing action $x^*$. Since $\epsilon=0$ and since both players could then only sample records of the other player playing $x^*$ both would continue to play $x^*$ as a best response for every period thereafter. So, I have shown that there exists a positive probability that a convention can be reached and locked into with positive probability in Case 1. 

\vskip24pt

\textit{Case 2}: Assume the players do not share a best reply to the most recent set of $s$ records. Since $\epsilon=0$ they play different actions as best replies in period $t+1$. Without loss of generality assume that in period $t+1$ player A played action 1 and player B played action 0. This means that:

$$(3) \sum\limits_{r=t-s+1}^{t} \frac{x^r_B}{s} > \alpha$$
and
$$(4) \sum\limits_{r=t-s+1}^{t} \frac{x^r_A}{s} < \beta$$

Note: this is a strict inequality since the players do not share a best reply in period $t+1$ after sampling the set of records: $(x^{t-s+1},...,x^{t})$.

\vskip12pt

\centering

Defining $k$ and $j$

\vskip6pt

\raggedright

Assume for the time being that player B continues to play action 0 for every period after period $t+1$. We know, since this is a coordination game, that if player A samples the most recent $s$ actions in every period that there will exist a period, let's call the first one period $t+1+k$, where player A will have action 0 as a best response.

\vskip12pt

Thus, $k$ is defined to be the smallest integer such that

$$ \sum\limits_{r=t-s+1+k}^{t+k} \frac{x^r_B}{s} \leq \alpha$$

Note that we assume that the record of $x^r_B$ is 0 for $r > t+1$. So, the sum of the records $x^r_B$ where $r > t+1$ is equal to 0 and drops out of this best response calculation. 

So, the above equation can be simplified to:

$$ (5) \sum\limits_{r=t-s+1+k}^{t+1} \frac{x^r_B}{s} \leq \alpha$$


Since $\alpha$ is positive we know the inequality holds when $k=s$, as that would force the right side of the equation to zero since the record of $x^{t+1}_B = 0$. Additionally, (3) implies that it does not hold when $k=0$. Thus, it is clear that for all histories and all $\alpha$, $k \in \{1,...,s\}$.

\vskip12pt

Likewise, let's assume for the time being that player A continues to play action 1 for every period after period $t+1$. We know, since this is a coordination game, that if player B samples the most recent $s$ actions in every period that there will exist a period, let's call the first one period $t+1+j$, where player B will have action 1 as a best response.

\vskip12pt

Thus, $j$ is defined to be the smallest integer such that

$$ \sum\limits_{r=t-s+1+j}^{t+j} \frac{x^r_A}{s} \geq \beta$$

Note that we assume that the record of $x^r_A$ is 1 for $r>t+1$. So, the sum of the records $\mfrac{x^r_A}{s}$ where $r>t+1$ is equal to $\mfrac{j-1}{s}$.

So, the above equation can be simplified to:

$$(6) \frac{j-1}{s}+\sum\limits_{r=t-s+1+j}^{t+1} \frac{x^r_A}{s} \geq \beta$$

 Since $\beta$ is less than 1 we know the inequality holds when $j=s$ since the right side of the equation equals 1 since the record of $x^{t+1}_A = 1$. Also, (4) tells us that it does not hold when $j=0$. Thus, it is clear that for all histories and all $\beta$, $j \in \{1,...,s\}$.
 
 \vskip6pt
 
 Note: allowing j = 0 is not problematic in (6) because we know $x^{t+1}_A = 1$, which will cancel out with $\mfrac{j-1}{s}$ leaving the expression to be equal to that in (4).

\vskip12pt

Now I will prove that for all periods after $t+1$ and before $t+1+k$ there is a positive probability that player A's best response is to play action 1. Likewise, I will prove that for all periods after $t+1$ and before $t+1+j$ there is a positive probability that player B's best response is to play action 0.

\vskip12pt

\centering

Proving best responses in periods $t+1+e \hspace{6pt} \forall e$ such that $0< e < k,j$

\vskip12pt

\raggedright
If $k = 1$ then there is nothing to prove with respect to player A as there is no integer between 0 and 1. So, I will prove that 1 can be a best response for player A in periods $t+1+e \hspace{6pt} \forall 0<e<k$ where $k > 1$.

\vskip6pt

Likewise, if $j = 1$ then there is nothing to prove with respect to player B as there is no integer between 0 and 1. So, I will prove that 0 can be a best response for player B in periods $t+1+e \hspace{6pt} \forall 0<e<j$ where $j > 1$.

\vskip6pt

Assume that in each period $t+1+e$ that players samples the most recent set of $s$ records. In this case, player A has action 1 as a unique best response if: 
$$(7) \hspace{12pt} \sum\limits_{r=t-s+1+e}^{t+e} \frac{x^r_B}{s} > \alpha$$

and player B has action 0 as a unique best response if:
$$(8) \hspace{12pt} \sum\limits_{r=t-s+1+e}^{t+e} \frac{x^r_A}{s} < \beta$$

Note that $j, k \leq s$ so $e \leq s-1$, and the restriction that $k,j > 1$ implicitly restricts $s$ to $s>1$. Because we are only concerned with cases where $1 \leq e \leq s-1$, and we know that $s>1$, we can dissect the summation into 2 parts without consequence:
$$(9) \hspace{12pt} \sum\limits_{r=t-s+1+e}^{t+e} \frac{x^r_i}{s} = \sum\limits_{r=t-s+1+e}^{t}\frac{x^r_i}{s}+\sum\limits_{r=t+1}^{t+e} \frac{x^r_i}{s}$$

So combining equations (7) and (9) where $i=B$, player A has action 1 as a unique best response if: 
$$(10) \sum\limits_{r=t-s+1+e}^{t}\frac{x^r_B}{s}+\sum\limits_{r=t+1}^{t+e} \frac{x^r_B}{s} > \alpha$$

and combining equations (8) and (9) where $i=A$, player B has action 0 as a unique best response if:
$$(11) \sum\limits_{r=t-s+1+e}^{t}\frac{x^r_A}{s}+\sum\limits_{r=t+1}^{t+e} \frac{x^r_A}{s} < \beta$$

\vskip18pt

Consider $1 \leq e<k$. Since $k$ is the smallest integer such that $$(5) \sum\limits_{r=t-s+1+k}^{t+1} \cfrac{x^r_B}{s} \leq \alpha$$

it follows that 

$$\forall e<k, \sum\limits_{r=t-s+1+e}^{t+1} \cfrac{x^r_B}{s} > \alpha$$ 

Further, since $x^{t+1}_B = 0$, and $e \leq s-1$, the expression 

$$\sum\limits_{r=t-s+1+e}^{t} \cfrac{x^r_B}{s} = \sum\limits_{r=t-s+1+e}^{t+1} \cfrac{x^r_B}{s} > \alpha$$

Since 

$$x_B^r \in [0,1] \hspace{8pt} \forall r, \sum\limits_{r=t+1}^{t+e} \cfrac{x^r_B}{s} \in [0,\mfrac{e}{s}]$$

So, because 

$$\sum\limits_{r=t+1}^{t+e} \cfrac{x^r_B}{s} \geq 0 \hspace{6pt}\text{and} \sum\limits_{r=t-s+1+e}^{t} \cfrac{x^r_B}{s} > \alpha$$

I get the condition (10):

$$\sum\limits_{r=t-s+1+e}^{t}\cfrac{x^r_B}{s}+\sum\limits_{r=t+1}^{t+e} \cfrac{x^r_B}{s} > \alpha$$

This condition means that for all integers $e$ such that $1 \leq e<k$ when sampling the most recent $s$ records in period $t+1+e$ that action 1 is a unique best response for player A.

\vskip18pt

The proof for player B playing 0 as a best response is similar to the one above:

\vskip6pt

Now consider $0 \leq e<j$. Since $j$ is the smallest integer such that 

$$(6) \cfrac{j-1}{s}+\sum\limits_{r=t-s+1+j}^{t+1} \cfrac{x^r_A}{s} \geq \beta$$ 

it follows that

$$\forall e<j, \cfrac{e-1}{s}+\sum\limits_{r=t-s+1+e}^{t+1} \cfrac{x^r_A}{s} < \beta$$

Further since $x^{t+1}_A = 1$, and $e \leq s-1$, the expression

$$\cfrac{e}{s}+\sum\limits_{r=t-s+1+e}^{t} \cfrac{x^r_A}{s} = \cfrac{e-1}{s}+\sum\limits_{r=t-s+1+e}^{t+1} \cfrac{x^r_A}{s} < \beta$$

Since 

$$x_A^r \in [0,1] \hspace{8pt} \forall r, \sum\limits_{r=t+1}^{t+e} \cfrac{x^r_A}{s} \in [0,\mfrac{e}{s}]$$

So, because 

$$\sum\limits_{r=t+1}^{t+e} \cfrac{x^r_A}{s} \leq \cfrac{e}{s} \hspace{6pt} \text{and} \hspace{6pt} \cfrac{e}{s}+\sum\limits_{r=t-s+1+e}^{t} \cfrac{x^r_A}{s} < \beta$$ 

I get the condition (11):

$$\sum\limits_{r=t-s+2+e}^{t}\cfrac{x^r_A}{s}+\sum\limits_{r=t+1}^{t+e} \cfrac{x^r_A}{s} < \beta$$

This condition means that for all integers $e$ such that $0 \leq e<j$ when sampling the most recent $s$ records in period $t+2+e$ that action 0 is a unique best response for player B.

\vskip12pt

There exists a positive probability that in each period $t+1+e$ both players can sample the most recent $s$ records. I have just shown that when sampling the most recent $s$ records in periods $0 < e < k$, player A best responds with action 1. When sampling the most recent $s$ records in periods $0 < e < j$, player B best responds with action 0. Assume both players do sample the most recent $s$ records in those periods. Since $\epsilon=0$ both players play their best response in those periods.

\vskip12pt

\centering

Proving coordination under different scenarios

\vskip6pt

\raggedright

I will now consider the three scenarios: $j<k$, $k<j$, and $j=k$.

\vskip12pt

First, $j<k$. 

\vskip6pt

For period $t+1+j$ both players have a positive probability of sampling the most recent $s$ records: $(x^{t-s+1+j},...,x^{t+j})$. Player B has a best response of action 1 in period $t+1+j$ if:
$$(12) \sum\limits_{r=t-s+1+j}^{t}\frac{x^r_A}{s}+\sum\limits_{r=t+1}^{t+j} \frac{x^r_A}{s} \geq \beta$$
Since $j<k$, I have already shown that there is a positive probability that player A has a unique best response of playing action 1 for all periods between $t+1$ and $t+1+j$ inclusive. So, $\sum\limits_{r=t+1}^{t+j} \cfrac{x^r_A}{s}=\cfrac{j}{s}$.

Which means 
$$\sum\limits_{r=t-s+1+j}^{t}\cfrac{x^r_A}{s}+\sum\limits_{r=t+1}^{t+j} \cfrac{x^r_A}{s}=\cfrac{j}{s}+\sum\limits_{r=t-s+1+j}^{t}\cfrac{x^r_A}{s}$$

Since we know $x^{t+1}_A = 1$, we know $$\cfrac{j}{s}+\sum\limits_{r=t-s+1+j}^{t}\cfrac{x^r_A}{s} = \cfrac{j-1}{s}+\sum\limits_{r=t-s+1+j}^{t+1}\cfrac{x^r_A}{s}$$

Using (6), we know:

$$\frac{j-1}{s}+\sum\limits_{r=t-s+1+j}^{t+1} \frac{x^r_A}{s} \geq \beta$$

which means that (12) holds and action 1 is a best response for Player B in period $t+1+j$.

\vskip6pt

Since both players can, with positive probability, sample the most recent $s$ records and have the same action as a best response in period $t+j+1$, I know by Case 1 a convention can be reached and locked into with positive probability when $j<k$.

\vskip18pt

Second, I consider $k<j$.

\vskip6pt

For period $t+1+k$ both players have a positive probability of sampling the most recent $s$ records: $(x^{t-s+1+k},...,x^{t+k})$. Player A has a best response of action 0 in period $t+1+k$ if:

$$(13) \sum\limits_{r=t-s+1+k}^{t}\frac{x^r_B}{s}+\sum\limits_{r=t+1}^{t+k} \frac{x^r_B}{s} \leq \alpha$$

Since $k<j$, I have already shown that player B has a unique best response of playing action 0 for all periods between $t+1$ and $t+1+k$ inclusive. So $\sum\limits_{r=t+1}^{t+k} \cfrac{x^r_B}{s}=0$.

Which means 

$$\sum\limits_{r=t-s+1+k}^{t}\frac{x^r_B}{s}+\sum\limits_{r=t+1}^{t+k} \frac{x^r_B}{s}=\sum\limits_{r=t-s+1+k}^{t}\frac{x^r_B}{s}$$

Since we know $x^{t+1}_B = 0$, we know

$$\sum\limits_{r=t-s+1+k}^{t}\frac{x^r_B}{s} = \sum\limits_{r=t-s+1+k}^{t+1}\frac{x^r_B}{s}$$

Using (5), we know:

$$\sum\limits_{r=t-s+1+k}^{t+1} \frac{x^r_B}{s} \leq \alpha$$

which means that (13) holds and action 0 is a best response for Player A in period $t+1+k$.

\vskip6pt

Since both players can, with positive probability, sample the most recent $s$ records and have the same action as a best response in period $t+k+1$, I know by Case 1 a convention can be reached and locked into with positive probability when $k<j$.

\vskip18pt

Third, I consider $k=j$.

\vskip6pt

In period $t+1+k$ player B can, with positive probability, sample the most recent set of $s$ records. Since $j=k$ I have already shown that playing action 1 is a best response for player B in this scenario.

\vskip6pt

Since $m>s$ and both $m$ and $s$ are integers, I know that $m \geq s+1$. So in period $t+1+k$ the records player A can sample from includes the most recent $s+1$ records: $(x^{t+k-s},...,x^{t+k})$. So in period $t+1+k$ player A can, with positive probability, sample the set of $s$ records: $(x^{t+k-s},...,x^{t+k-1})$. Note that these are the same records sampled in the previous period, $t+k$, by player A which gave the unique best response of playing action 1.

\vskip6pt

So, there exists a positive probability that both players share a best response in period $t+1+k$. As consequence, I can apply Lemma 1 here which shows that there exists a positive probability that both players play action 1 as a best reply for each period $e \geq t+1+k$. If this happens, then after period $t+k+m$ the entire memory is filled with both players playing action 1. Since $\epsilon=0$ and since both players could then only sample records of the other player playing action 1 both would continue to play 1 as a best response for every period thereafter. So I have shown that there exists a positive probability that a convention can be reached and locked into when $j=k$. 

\vskip12pt

Thus, I have exhausted all three scenarios: $j<k$, $k<j$, and $j=k$ and shown that a convention can be reached with a positive probability in Case 2.

\vskip12pt

Since I have shown that a convention can be reached with positive probability in both Case 1 and Case 2 I have proven that from any initial state when $s<m$ and $\epsilon=0$ a convention can be reached with positive probability and lock in. Since a convention can be reached from any arbitrary state and since conventions are absorbing states we know that as $T \rightarrow \infty$ that $h^T$, the state at time $T$ converges with probability one to a convention in and locks in. $\blacksquare$





\end{document}