\documentclass{article}
\usepackage[utf8]{inputenc}
\usepackage{dsfont}
\usepackage{scrextend}
\usepackage{multirow,array}
\begin{document}
\centering
\Large
\textbf{Convergence to a Convention in a 2x2 Coordination Game with Adaptive Learning}
\vskip0pt
Ethan Holdahl

University of Oregon

October 2019

\vskip24pt

\centering
Coordination Game G
\vskip6pt
    \begin{tabular}{cc|c|c|}
      & \multicolumn{1}{c}{} & \multicolumn{2}{c}{Player B}\\
      & \multicolumn{1}{c}{} & \multicolumn{1}{c}{$0$}  & \multicolumn{1}{c}{$1$} \\\cline{3-4}
      \multirow{2}*{Player A}  & $0$ & ${a_{11},b_{11}}^*$ & $a_{12},b_{12}$ \\\cline{3-4}
      & $1$ & $a_{21},b_{21}$ & ${a_{22},b_{22}}^*$ \\\cline{3-4}
    \end{tabular}

\vskip24pt

\raggedright

\textbf{Theorem.} Let G be a 2x2 coordination game, and let $P^{m,s,\epsilon}$ be adaptive learning with memory $m$, sample size $s$, and error rate $\epsilon$. Let $\alpha$ be the smallest probability that Player B plays column $1$ such that Player A's best response is playing row $1$. Likewise, Let $\beta$ be the smallest probability that Player A plays row $1$ such that Player B's best response is playing column $1$.

\vskip12pt
If $s < m$ then from any initial state, the unperturbed process $P^{m,s,0}$ converges with probability one to a convention and locks in.

\vskip12pt

\textbf{Proof.} Let $h = (x^{t-m+1},...,x^{t})$ be an arbitrary state. There exists a positive probability that both players sample the set of records $(x^{t-s+1},...,x^{t})$ in period $t+1$ and since $\epsilon=0$ each plays a best reply. Since there exists a positive probability that both coordinate when either does not have a unique best reply assume they do coordinate in that instance.

\vskip12pt

\textit{Case 1}: Both players best replies coordinate in period $t+1$. Without loss of generality I will say they both played 1 $(x^{t+1}_A=x^{t+1}_B=1)$. Note that since $x_i \in [0,1]$ the proportion of times that player $i$ played $1$ in $s$ records, say from $t-s+1$ to $t$ inclusively, is simply $\sum\limits_{r=t-s+1}^{t} \frac{x_i^r}{s}$. Also note since they both played 1 as best reply that means that 
$$\text{(1)} \hspace{12pt} \sum\limits_{r=t-s+1}^t \frac{x^r_A}{s} \geq \beta$$
and 
$$\sum\limits_{r=t-s+1}^t \frac{x^r_B}{s} \geq \alpha$$

In the next period, $t+2$, there exists a positive probability that both players sample the set of records $(x^{t-s+2},...,x^{t+1})$ and since $\epsilon=0$ each plays a best reply again with both choosing to play 1 if there is not a unique best reply. Looking at player B's decision, they will play 1 if
$$\text{(2)} \hspace{12pt} \sum\limits_{r=t-s+2}^{t+1} x^r_A \geq \beta*s$$
Note that
$$x^{t-s+1}_A+\sum\limits_{r=t-s+2}^{t+1} x^r_A = x^{t+1}_A+\sum\limits_{r=t-s+1}^t x^r_A$$
Since 
$x^{t+1}_A=1$ and $x^{t-s+1} \leq 1$, we get
$$\text{(3)} \hspace{12pt} \sum\limits_{r=t-s+2}^{t+1} x^r_A \geq \sum\limits_{r=t-s+1}^t x^r_A $$
By transitivity, using (1) and (3) we get our condition (2):
$$\sum\limits_{r=t-s+2}^{t+1} x^r_A \geq \sum\limits_{r=t-s+1}^t x^r_A \geq \beta*s$$

So a best response for player B is to play column 1. The same logic can be used to show that a best response for player A is to play row 1.

\vskip12pt

This process of sampling the most recent $s$ records for both players can be repeated until period $t+m$ with positive probability at which point the entire memory is comprised of records of each player coordinating at a Nash Equilibria. At such a point, barring errors, both players will continue to coordinate forever and by definition we are locked in at a convention.

This case proves that if after sampling the most recent s records for both players they both best reply with the same action a convention can be reached with positive probability.

\vskip12pt

\textit{Case 2}: Assume the players did not coordinate in period $t+1$. In each period $E \in [t+2,...,t+s+1]$ there exists a positive probability that both players sample the set of records $(x^{E-s},...,x^{E-1})$. Since $\epsilon=0$ each player will play a best reply. Since there exists a positive probability that both coordinate when either does not have a unique best reply assume they do coordinate in that instance.

At this point, the most recent $s$ records are either perfectly miscoordinated or have coordinated at some record, $C \in [t+2,t+s+1]$. This means that at period $C$ both players sampled the set of records $(x^{C-s},...,x^{C-1})$ and shared the same best response. This is exactly Case 1 which means that a convention can be reached with positive probability.

I have only left to show that a convention can be reached with positive probability if the records $(x^{t+2},...,x^{t+s+1})$ are perfectly miscoordinated.

Without loss of generality assume that in period $t+s+1$ player A played row 1 and player B played column 0. This means that 
$$\sum\limits_{r=t+1}^{t+s} \frac{x^r_B}{s} > \alpha$$
and
$$\sum\limits_{r=t+1}^{t+s} \frac{x^r_A}{s} < \beta$$
Note: this is a strict inequality since I specified that if the best reply was not unique that the players would coordinate.

From this point forward if the lower bound on a summation is larger than the upper bound on a summation we will consider the value of the sum to be 0.

Define $k$ as the smallest integer such that 
$$(4) \hspace{12pt} \sum\limits_{r=t+1+k}^{t+s} \frac{x^r_B}{s} \leq \alpha$$
Clearly, $k \in [1,s]$. This means that if player B plays column 0 for every period from $t+s+1$ up to and including $t+s+k$ and player A samples the most recent $s$ set of records for every period up to $t+s+k+1$ that player A's unique best response will be to play row 1 from period $t+s+1$ up to $t+s+k$, inclusive, and one of player A's best responses in period $t+s+k+1$ will be to play row 0.

Define $j$ as the smallest integer such that
$$(5) \hspace{12pt} \frac{j}{s}+\sum\limits_{r=t+1+j}^{t+s} \frac{x^r_A}{s} \geq \beta$$

Clearly, $j \in [1,s]$. This means that if player A plays row 1 for every period from $t+s+1$ up to and including $t+s+j$ and player B samples the most recent $s$ set of records for every period up to $t+s+j+1$ that player B's unique best response will be to play column 0 from period $t+s+1$ up to $t+s+j$, inclusive, and one of player B's best responses in period $t+s+j+1$ will be to play row 1.
\vskip6pt

If $j \neq k$ then if in each period $E \in [t+s+2,..,t+s+Max(j,k)]$ both players sample the set of records $(x^{E-s},...,x^{E-1})$ the best response of each player can coordinate in period $t+s+Min(j,k)+1$. Specifically, if $j<k$ then a best response for both players will be to play 1 in period $t+s+j+1$ and if $k<j$ then a best response for both players will be to play 0 in period $t+s+k+1$.

In both $k<j$ and $j<k$ we have exactly Case 1 which means that a convention can be reached with  positive probability.
\vskip6pt

So we have left to consider the case where $j=k$.

So if in each period $E \in [t+s+2,..,t+s+k]$ both players sample the set of records $(x^{E-s},...,x^{E-1})$, each will play their unique best response with player A playing row 1 and player B playing column 0 as described under equations (4) and (5). In period $t+s+k+1$ since $m>s$ there exists a positive probability that player A samples the most recent $s+1$ set of records minus the most recent record, $t+s+k$. So they sample $(x^{t+k},...,x^{t+s+k-1})$, which is the same set they sampled last period which gives a player A a best response of playing row 1. In the same period, $t+s+k+1$, there exists a positive probability that player B samples the most recent set of $s$ records: $(x^{t+k+1},...,x^{t+s+k})$. Player B's best response in this sample is column 1.
\vskip6pt

In periods $E \in [t+s+k+2,..,t+2s+k]$, player B can continue to sample the most recent set of $s$ records with positive probability. Given that player A plays row 1 in periods $[t+s+k+1,...,t+2s+k-1]$, player B's best response will be to play column 1. This proof is essentially identical to that in Case 1. In periods $E \in [t+s+k+2,..,t+2s+k]$, player A can continue to sample the most recent set of $s+1$ records minus the $t+s+k$th record: $(x^{E-s-1},...,x^{t+s+k-1},x^{t+s+k+1},...,x^{E-1})$ with positive probability. Given that player B plays column 1 in periods $[t+s+k+1,...,t+2s+k-1]$ player A's best response is to play row 1.

Since $k$ is the smallest integer such that (4) is satisfied it follows that for all $y$ such that $y \in [0,s]$.
$$\frac{y}{s}+\sum\limits_{r=t+1+(k-1)+y}^{t+s} \frac{x^r_B}{s} > \alpha$$
This represents the decision making process for player A when counting the most recent $s+1$ records minus the $t+s+k$th record and where $y$ is the total number of records after the $t+s+k$th record for which player B plays column 1 every period. Since the equality holds for all $y$ up to $y=s$ we know that player A's best response will be to play row 1 for every period after $t+s$th period.

Since each player plays 1 in period $t+k+s+1$ that satisfies the requirement necessary for the other player to have a best response of 1 in the following period. This process repeats itself until period $t+k+2s$. 

In period $t+2s+k+1$ both players can sample the set of records $(x^{t+k+s+1},...,x^{t+2s+k})$. Here both players have played 1 for every record in the sample so each player's best response is to play 1. Since both players sampled the most recent $s$ records and had the same best response this is Case 1 which means that a convention can be reached with positive probability.
\vskip12pt
Since a convention can be met for any $s<m$ from any initial state we know as $t \rightarrow \infty$ the unperturbed process $P^{m,s,0}$ converges with probability one to a convention and locks in.



\end{document}